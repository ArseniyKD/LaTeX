\documentclass{beamer}

\usetheme{JuanLesPins}


\title{STAT 151 Midterm Review}
%\subtitle{From LHSA Academics Comittee}
\author{Arseniy Kouzmenkov}
\institute{University of Alberta}

\begin{document}

    \begin{frame}
        \titlepage
    \end{frame}

     \section{Introduction to Presentation}
    
    \begin{frame}
        \frametitle{Outline}
        \tableofcontents
    \end{frame}

    \begin{frame}
        \frametitle{Introduction to Presentation}
        This presentation was made by Arseniy Kouzmenkov for the LHSA.\newline \pause

        What this presentation will help you with: 
        \begin{itemize}
            \item Point out gaps in understanding. \pause
            \item Help you figure out what topics to focus on. \pause
            \item Teach you where to look in the formula sheet. \pause
            \item Give some sample questions to go through. \pause
        \end{itemize}

        What this presentation will \textbf{not} help you with:
        \begin{itemize}
            \item Teach you new material, this is a review after all. \pause
            \item Learn all the material from scratch. \pause
            \item Substitute the Sample Midterm or your professor's notes \pause
        \end{itemize}

        Now that the expectations are set, let's begin!         
    \end{frame}

    \section{Types of Samples}

    \begin{frame}
        \frametitle{Types of Samples}
        \begin{itemize}
            \item Simple Randome Sample:
            \begin{itemize}
                \item[] A random selection from the whole population.
                \newline e.g.: Picking a random hockey player from all hockey players. \newline 
            \end{itemize}
            \pause
            \item Stratified Random Sample:
            \begin{itemize}
                \item[] Divide populations into groups, then randomly pick from those groups. \newline
                e.g.: Dividing hockey players into age groups and then randomly picking a player 
                    from each age group. \newline
            \end{itemize}
            \pause
            \item Systematic Sampling:
            \begin{itemize}
                \item[] Selection from the whole population based on a specific method. \newline
                    e.g.: Picking every 9\textsuperscript{th} hockey player for a survey.
            \end{itemize}
        \end{itemize}
    \end{frame}

    \subsection{Biased}

    \begin{frame}
        \frametitle{Types of Samples -- Biased}
        \begin{itemize}
            \item Voluntary Response Sampling:
            \begin{itemize}
                \item[] A sample consisting entirely of volunteers. \newline
                e.g.: Residence Services asking for Lister's feedback about the meal plan in their newsletter. 
                \newline
            \end{itemize}
            \item Convenience Sampling:
            \begin{itemize}
                \item[] A sample consisting of easy to reach responses. \newline
                e.g.: Interviewing only people on your floor about topics concerning the entire tower.
            \end{itemize}
        \end{itemize}
    \end{frame}

    \section{Types of Studies}

    \begin{frame}
        \frametitle{Types of Studies}
        \begin{itemize}
            \item Observational:
            \begin{itemize}
                \item[] Researchers observe a group. You cannot infer cause and effect; however, you can
                make population inferences in this case. \newline
                e.g.: Looking at roommates who have the flu and seeing it's progression. \newline \pause
            \end{itemize}
            \item Experimental:
            \begin{itemize}
                \item[] Researchers change a factor to study how it affects a group. You may infer cause 
                and effect relationships. \newline
                e.g.: Giving the sick roommates flu medication and studying it's effects.
            \end{itemize}
        \end{itemize}
    \end{frame}

    \subsection{Types of Obserational Studies}

    \begin{frame}
        \frametitle{Types of Observational Studies}
        \begin{itemize}
            \item Retrospective:
            \begin{itemize}
                \item[] Researchers study a group from a past time period and study how it compares
                to other similar groups \newline
                e.g.: Reading a history book and studying how the ``black death" spread throughout 
                Europe. \newline \pause
            \end{itemize}
            \item Prospective:
            \begin{itemize}
                \item[] Researchers set certain criteria and follow a group that fulfills the criteria
                 to see how it may affect outcomes.\newline
                e.g.: Following some Lister residents on campus to see if they eat outside of the cafeteria.
            \end{itemize}
        \end{itemize}
    \end{frame}

    \section{Quartiles}

    \begin{frame}
        \frametitle{Quartiles}

    \end{frame}

    \section{Mean, Median, IQR, and Standard Deviation}
    
    \begin{frame}
        \frametitle{Mean, Median, IQR, and Standard Deviation}
    \end{frame}

    \section{Normal Distributions}

    \begin{frame}
        \frametitle{Normal Distributions}
    \end{frame}

    \section{Expected Value}

    \begin{frame}
        \frametitle{Expected Value}
    \end{frame}
    
    \section{Dependent Probability}

    \begin{frame}
        \frametitle{Dependent Probability}
    \end{frame}

    \section{Sampling Distributions}

    \begin{frame}
        \frametitle{Sampling Distributions}
    \end{frame}

    \section{Sample Questions}

    \begin{frame}
        \frametitle{Sample Questions}
    \end{frame}


\end{document}
